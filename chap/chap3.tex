\chapter{自动加样系统硬件系统设计方案的确定}
\section{电机类型的选择}
自动加样系统在加样过程中进行多次重复加样,位置重复性要求好;加样系统必须具有一定的效率,短时间内完成较多的加样任务,速度响应要求要好;加样过程中,为保证测试结果的准确度取样要准,运动平稳性和精度要求高。电力驱动所用的驱动器通常有直流伺服电机、交流伺服电机、步进电机。对于伺服电机,其主要特点是马达或执行机构上装有传感器,传感器将命令执行结果经由伺服放大器传回控制中心,控制中心在比较执行结果和指令值之后再对执行结果进行反馈纠正以不断减小运动执行误差。可以实现转速可以精确控制,速度控制范围广,除了可以进行稳定平顺等速运转之外,还可以根据需求随时变更速度。在极低速度也可以稳定转动。然而,一般伺服电机采取的控制方式是闭环控制,闭环系统中机械结构的惯性、传动误差、变形问题使得控制极其复杂和困难。而步进电机不需要运转量检知器或编码器,且由脉冲信号控制执行机构运动的距离和速度,不需要位置检出和速度检出的回授元件,所以步进电机可正确地依比例跟随脉冲信号而转动,因此就能达成精确的位置和速度控制,且稳定性佳,例外可以通过位置传感器(如限位开关)来辅助位置和速度的控制。综合控制成本和控制效果来看,本文选择步进电机作为$X$、$Y$、$Z$方向平动的驱动器。












