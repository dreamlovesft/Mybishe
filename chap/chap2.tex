\chapter{自动加样系统机械臂的设计}
%自动加样系统的一个基本并且关键功用就是在空间内实现任%意位置的准确定位。因此,自动加样系统的加样机械臂的设%计十分重要和关键。本部分完成对其系统结构的设计。
\section{加样机械臂结构类型的选定}
机械臂结构类型多种多样,不同工作场景所使用的机械臂结构形式千差万别。工业上广泛使用的机械臂结构主要有以下几种:

1.铰接式结构:该设计由多个旋转接头和工作臂组成,从简单的双链接结构到多(5-8个以上)相互作用的铰接系统。一般地,该结构在笛卡尔坐标系下能够实现三个转动自由度。一般情况下,铰接结构越多,工作范围越大,机械臂运动也就越精确。铰接式机械臂通常用于装配,压铸,气焊,及喷涂等工作场景。

2.笛卡尔式结构:该结构在笛卡尔坐标系下一般有三个平动自由度,它能实现机械臂在空间内沿着X,Y,Z方向平动从而到达坐标系中任意位置。笛卡尔式机械臂通常用于装载(卸载)工件,印刷电子电路板,材料表面处理等工作场景。该结构提供的工作空间大(可用于其它用途)、控制系统简单。

3.圆柱式结构:该结构在笛卡尔坐标系统能实现沿着X,Y方向平动和绕Z方向转动,机器臂可以移动到由圆柱体描述的体积内的任意位置。该结构在垂直方向上节省了空间,具有良好的刚性能承受较大的有效载荷,但由于其结构限制不能实现360度转动。

4.极坐标式结构:该结构在笛卡尔坐标系下一般能够实现两个旋转自由度和一个平动自由度。相比于圆柱式结构,它在空间内能实现360度的转动,并且在水平方向上的工作距离更长。

根据加样系统的功用和动作流程,加样机械臂要能够到达三维空间中的任意位置;加样多孔板需要放置在工作台上,要求工作台空间足够大;加样液在移送过程中不能有滴漏,要求加样机械臂运动平稳;取样、释样过程不能发生样品(试剂)出错现象,要求加样机械臂运动精度高。考虑以上要素,综合对比上述四种机械臂结构本文选用笛卡尔式结构,该结构能够提供较大的工作空间,控制易于实现,运动平稳且精度高。
\section{驱动方式的选定}
对于机械臂的驱动方式,本文首先考虑使用最为广泛的三种驱动方式:液压驱动、气压驱动、电力驱动,以保证加样机械臂运动的平稳性和位置精确度.

加样机械臂的运动状态是由其驱动系统决定的,它一般由两个部分组成:驱动机构和传动机构。驱动机构其实质可以视为一种能量转换装置,它能将液压能、气动能量、电能等转化为加样机械臂的动能,并通过传动机构使得机械臂动作。根据驱动原理的不同,可以将其分为电力驱动、液压驱动和气压驱动三种类型。

电力驱动:该方式利用电动机产生的力矩和力驱动机械臂进行动作,其具有速度调节范围广、控制精度高、运动平稳性高的特点,具有良好的的控制性和环境适应性。常用的电机驱动器包括直流伺服电机、交流伺服电机、步进电机等。

液压驱动:该方式利用液体产生的压力驱动机械臂进行动作,其具有动力大,无级调速,能实现高速的精度控制等特点。但供油回油等附加元件使其系统体积大,成本高,维修困难,易发生液压油的泄漏而污染环境、稳定性差。该驱动方式多用于矿山机械,重型设备等。

气压驱动:该方式利用气体产生的压力驱动机械臂进行动作,其具有响应快清洁无害,易维护等特点。但其运动稳定性差,功率小,噪声大、难以实现较高精度的位置和速度控制,多用于小功率驱动、运动精度要求低的场合。

自动加样系统实现定量甚至是微量加样,要求其运动平稳性好,位置精度高,运动重复精度好;其部件模块不大、负载小;并且,自动加样系统一般多用于医院、科研机构等对噪声要求严苛的地方。比较上述三种驱动方式,本文选用电力驱动。
\section{传动方式选定}
加样机械臂传动方式对加样精度、稳定性和相应的快速性都产生重要影响。机械系统中传动机构多种多样,本文初步选定将步进电机的回转运动转换成直线运动的滚珠丝杠螺母传动和同步带(工作原理,优点减轻驱动系统重量,简化了其结构)。

滚珠丝杠螺母传动机构与普通丝杠螺母副相似,但其在丝杠和螺母之间装有滚珠。丝杠转动的同时带动滚珠在螺纹滚道里转动,转动的滚珠又将运动传递给螺母使其平动。滚珠丝杠螺母传动机构的传递的效率高,摩擦损失小,传动精度高但是其结构复杂、难以加工、制造成本高;并且滚珠丝杠螺母机构需要润滑和密封机构以提高其工作效率和寿命。

同步带传动是带传动和齿轮传动的有机结合体,兼备二者的优点。同步带的内周表面制有等距的距齿。工作时,利用带与带轮齿之间的啮合关系传递运动\supercite{bib7}。同步带的传动比准确且恒定,带的缓冲吸振特性使其传动平稳,结构紧凑,不需要密封和润滑维护方便,并且重量小,传动能量利用效率高,刚度影响因素少。

加样机械臂的传动比要求较高,传动过程不能发生干涉现象,尽量降低回程误差以提高传动精度,故选取同步带作为传动机构。












































